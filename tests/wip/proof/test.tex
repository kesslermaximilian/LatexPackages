\documentclass[german]{article}

\usepackage{mkessler-proof}
\usepackage{mkessler-fancythm}
\usepackage{mkessler-hypersetup}
\usepackage{parskip}

\begin{document}
\section{test}

\begin{theorem}\label{thm:krass}
    Man sollte nach Würzburg fahren.
\end{theorem}

\begin{rproof}{thm:krass}
    \begin{claim}\label{cl:qed}
       Der QED ist toll.
   \end{claim}
   Um \autoref{cl:qed} zu beweisen, brauchen wir zunächst ein Lemma. 
\end{rproof}

\begin{lemma}\label{lm:krass}
    Mathevereine sind krass.
\end{lemma}

\begin{rproof}{lm:krass}
    \begin{claim}
        Mathe ist cool.
    \end{claim}
    \begin{subproof}
        trivial.
    \end{subproof}
    Damit folgt nun das Lemma.
\end{rproof}

Nun kommen wir wieder zurück zum eigentlichen Beweis:

\begin{rproof}{thm:krass}
    \begin{claim}\label{cl:würzburg}
        Der QED macht ein Seminar in Würzburg
    \end{claim}
    \begin{subproof}
        Zu prüfen in der DB. Fakt!
    \end{subproof}
Aus \autoref{cl:qed}   und \autoref{cl:würzburg} folgt nun die Aussage. 
\end{rproof}
test
\end{document}
